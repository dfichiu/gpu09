\documentclass{report}

\usepackage{amsmath}
\usepackage{graphicx}

\begin{document}
\title{GPU Exercise 02 - Group 09}
\maketitle


\section*{2.1 Raw Kernel Startup Time}

The asynchronous kernel statup time (ASKST) of the original program is 0.05 us.

The synchronous kernel startup time (SKST) is 0.06 us.

The ASKST is the time it takes the CPU to launch a kernel on the GPU without waiting for the
kernel to start or complete. In contrast to that, the SKST is the time it takes for the CPU to issue a command to the GPU and wait until the GPU
completed executing the kernel (i.e., overhead from CPU-GPU synchronization).

Therefore, to measure the SKST in nullKernelAync.cu, we moved the barrier synchornization inside the for-loop.

For the last point, we vary the grid size (numBlocks) from 1 to 16384 in increments of 1024 and the block size (threadsPerBlock) from 1 to 1024 by using the powers of two.





\section*{2.3 PCIe Data Movements}
For measuring the memory allocation of pageable vs pinned memory, did we choose a vector size of 1kB, as this seems to be the maximum capacity of the pinned memory.
The pageable memory allocation is with 0.00us quite fast compared to the memory allocation of pinned memory with 3.03 us.



Pinned Memory
\begin{center}
    \begin{tabular}{ c c c }
        & D2H & H2D \\ 
     1kb & 7.84 us & 5.76 us \\ 
     5kb & 8.49 us & 6.18 us \\  
     10kb &  9.24  us & 6.34 us \\ 
     50kb & 14.08 us & 11.61 us \\  
     100kb &  20.85   us & 19.64 us \\ 
     500kb & 77.35  us &  73.68 us \\    
     1mb &150.47 us & 141.75 us \\ 
     10mb & 1085.44 us &  908.99 us \\ 
     100mb &  10750.17   us & 8568.39 us \\ 
     1gb & 184770.24 us &  87286.34 us \\ 
    \end{tabular}
    \end{center}

    INSERT PLOT?

    putting data onto the device is faster than moving it back to the host, but this difference is negligible. 
    in the datarsize range below 10kb the duration seems to be constant, but with duration increases linearly with the transfer size.

